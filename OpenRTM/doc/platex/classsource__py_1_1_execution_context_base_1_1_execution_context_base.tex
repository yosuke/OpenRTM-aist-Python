\section{クラス ExecutionContextBase}
\label{classsource__py_1_1_execution_context_base_1_1_execution_context_base}\index{source\_\-py::ExecutionContextBase::ExecutionContextBase@{source\_\-py::ExecutionContextBase::ExecutionContextBase}}
ExecutionContext用基底クラス  


\subsection*{Public メソッド}
\begin{CompactItemize}
\item 
def {\bf tick}
\begin{CompactList}\small\item\em ExecutionContextの処理を進める(サブクラス実装用) \item\end{CompactList}\end{CompactItemize}


\subsection{説明}
ExecutionContext用基底クラス 



\footnotesize\begin{verbatim}
\end{verbatim}
\normalsize


ExecutionContextの基底クラス。

\begin{Desc}
\item[から:]0.4.0 \end{Desc}


 ExecutionContextBase.py の 32 行で定義されています。

\subsection{関数}
\index{source\_\-py::ExecutionContextBase::ExecutionContextBase@{source\_\-py::ExecutionContextBase::ExecutionContextBase}!tick@{tick}}
\index{tick@{tick}!source_py::ExecutionContextBase::ExecutionContextBase@{source\_\-py::ExecutionContextBase::ExecutionContextBase}}
\subsubsection{\setlength{\rightskip}{0pt plus 5cm}def tick ( {\em self})}\label{classsource__py_1_1_execution_context_base_1_1_execution_context_base_125397433d5d1d506776ec7982be4b92}


ExecutionContextの処理を進める(サブクラス実装用) 

ExecutionContextの処理を1周期分進める。\par
 ※サブクラスでの実装参照用

\begin{Desc}
\item[引数:]
\begin{description}
\item[{\em self}]\end{description}
\end{Desc}


 ExecutionContextBase.py の 50 行で定義されています。