\section{uuid.py File Reference}
\label{uuid_8py}\index{uuid.py@{uuid.py}}
{\tt \#include \char`\"{}ctypes.py\char`\"{}}\par
{\tt \#include \char`\"{}ctypes/util.py\char`\"{}}\par
\subsection*{Classes}
\begin{CompactItemize}
\item 
class {\bf UUID}
\end{CompactItemize}
\subsection*{Functions}
\begin{CompactItemize}
\item 
{\bf \_\-ifconfig\_\-getnode} ()
\item 
{\bf \_\-ipconfig\_\-getnode} ()
\item 
{\bf \_\-netbios\_\-getnode} ()
\item 
{\bf \_\-unixdll\_\-getnode} ()
\item 
{\bf \_\-windll\_\-getnode} ()
\item 
{\bf \_\-random\_\-getnode} ()
\item 
{\bf getnode} ()
\item 
{\bf uuid1} (node=None, clock\_\-seq=None)
\item 
{\bf uuid3} (namespace, name)
\item 
{\bf uuid4} ()
\item 
{\bf uuid5} (namespace, name)
\end{CompactItemize}


\subsection{Function Documentation}
\index{uuid.py@{uuid.py}!_ifconfig_getnode@{\_\-ifconfig\_\-getnode}}
\index{_ifconfig_getnode@{\_\-ifconfig\_\-getnode}!uuid.py@{uuid.py}}
\subsubsection{\setlength{\rightskip}{0pt plus 5cm}\_\-ifconfig\_\-getnode ()}\label{uuid_8py_a0}


Get the hardware address on Unix by running ifconfig. \index{uuid.py@{uuid.py}!_ipconfig_getnode@{\_\-ipconfig\_\-getnode}}
\index{_ipconfig_getnode@{\_\-ipconfig\_\-getnode}!uuid.py@{uuid.py}}
\subsubsection{\setlength{\rightskip}{0pt plus 5cm}\_\-ipconfig\_\-getnode ()}\label{uuid_8py_a1}


Get the hardware address on Windows by running ipconfig.exe. \index{uuid.py@{uuid.py}!_netbios_getnode@{\_\-netbios\_\-getnode}}
\index{_netbios_getnode@{\_\-netbios\_\-getnode}!uuid.py@{uuid.py}}
\subsubsection{\setlength{\rightskip}{0pt plus 5cm}\_\-netbios\_\-getnode ()}\label{uuid_8py_a2}


Get the hardware address on Windows using Net\-BIOS calls. See {\tt http://support.microsoft.com/kb/118623} for details. \index{uuid.py@{uuid.py}!_random_getnode@{\_\-random\_\-getnode}}
\index{_random_getnode@{\_\-random\_\-getnode}!uuid.py@{uuid.py}}
\subsubsection{\setlength{\rightskip}{0pt plus 5cm}\_\-random\_\-getnode ()}\label{uuid_8py_a5}


Get a random node ID, with eighth bit set as suggested by RFC 4122. \index{uuid.py@{uuid.py}!_unixdll_getnode@{\_\-unixdll\_\-getnode}}
\index{_unixdll_getnode@{\_\-unixdll\_\-getnode}!uuid.py@{uuid.py}}
\subsubsection{\setlength{\rightskip}{0pt plus 5cm}\_\-unixdll\_\-getnode ()}\label{uuid_8py_a3}


Get the hardware address on Unix using ctypes. \index{uuid.py@{uuid.py}!_windll_getnode@{\_\-windll\_\-getnode}}
\index{_windll_getnode@{\_\-windll\_\-getnode}!uuid.py@{uuid.py}}
\subsubsection{\setlength{\rightskip}{0pt plus 5cm}\_\-windll\_\-getnode ()}\label{uuid_8py_a4}


Get the hardware address on Windows using ctypes. \index{uuid.py@{uuid.py}!getnode@{getnode}}
\index{getnode@{getnode}!uuid.py@{uuid.py}}
\subsubsection{\setlength{\rightskip}{0pt plus 5cm}getnode ()}\label{uuid_8py_a6}


Get the hardware address as a 48-bit integer. The first time this runs, it may launch a separate program, which could be quite slow. If all attempts to obtain the hardware address fail, we choose a random 48-bit number with its eighth bit set to 1 as recommended in RFC 4122. \index{uuid.py@{uuid.py}!uuid1@{uuid1}}
\index{uuid1@{uuid1}!uuid.py@{uuid.py}}
\subsubsection{\setlength{\rightskip}{0pt plus 5cm}uuid1 (node = {\tt None}, clock\_\-seq = {\tt None})}\label{uuid_8py_a7}


Generate a {\bf UUID}{\rm (p.\,\pageref{classUUID})} from a host ID, sequence number, and the current time. If 'node' is not given, {\bf getnode()}{\rm (p.\,\pageref{uuid_8py_a6})} is used to obtain the hardware address. If 'clock\_\-seq' is given, it is used as the sequence number; otherwise a random 14-bit sequence number is chosen. \index{uuid.py@{uuid.py}!uuid3@{uuid3}}
\index{uuid3@{uuid3}!uuid.py@{uuid.py}}
\subsubsection{\setlength{\rightskip}{0pt plus 5cm}uuid3 (namespace, name)}\label{uuid_8py_a8}


Generate a {\bf UUID}{\rm (p.\,\pageref{classUUID})} from the MD5 hash of a namespace {\bf UUID}{\rm (p.\,\pageref{classUUID})} and a name. \index{uuid.py@{uuid.py}!uuid4@{uuid4}}
\index{uuid4@{uuid4}!uuid.py@{uuid.py}}
\subsubsection{\setlength{\rightskip}{0pt plus 5cm}uuid4 ()}\label{uuid_8py_a9}


Generate a random {\bf UUID}{\rm (p.\,\pageref{classUUID})}. \index{uuid.py@{uuid.py}!uuid5@{uuid5}}
\index{uuid5@{uuid5}!uuid.py@{uuid.py}}
\subsubsection{\setlength{\rightskip}{0pt plus 5cm}uuid5 (namespace, name)}\label{uuid_8py_a10}


Generate a {\bf UUID}{\rm (p.\,\pageref{classUUID})} from the SHA-1 hash of a namespace {\bf UUID}{\rm (p.\,\pageref{classUUID})} and a name. 