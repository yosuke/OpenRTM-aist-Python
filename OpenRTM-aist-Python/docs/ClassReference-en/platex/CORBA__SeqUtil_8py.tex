\section{CORBA\_\-Seq\-Util.py File Reference}
\label{CORBA__SeqUtil_8py}\index{CORBA_SeqUtil.py@{CORBA\_\-SeqUtil.py}}
CORBA sequence utility template functions. 

\subsection*{Functions}
\begin{CompactItemize}
\item 
{\bf for\_\-each} (seq, f)
\begin{CompactList}\small\item\em CORBA sequence helper template functions\-Apply the functor to all CORBA sequence elements. \item\end{CompactList}\item 
{\bf find} (seq, f)
\begin{CompactList}\small\item\em Return the index of CORBA sequence element that functor matches. \item\end{CompactList}\item 
{\bf push\_\-back} (seq, elem)
\begin{CompactList}\small\item\em Push the new element back to the CORBA sequence. \item\end{CompactList}\item 
{\bf push\_\-back\_\-list} (seq1, seq2)
\item 
{\bf insert} (seq, elem, index)
\begin{CompactList}\small\item\em Insert the element to the CORBA sequence. \item\end{CompactList}\item 
{\bf front} (seq)
\begin{CompactList}\small\item\em Get the front element of the CORBA sequence. \item\end{CompactList}\item 
{\bf back} (seq)
\begin{CompactList}\small\item\em Get the last element of the CORBA sequence. \item\end{CompactList}\item 
{\bf erase} (seq, index)
\begin{CompactList}\small\item\em Erase the element of the specified index. \item\end{CompactList}\item 
{\bf erase\_\-if} (seq, f)
\item 
{\bf clear} (seq)
\begin{CompactList}\small\item\em Erase all the elements of the CORBA sequence. \item\end{CompactList}\end{CompactItemize}


\subsection{Detailed Description}
CORBA sequence utility template functions. 

\begin{Desc}
\item[Date:]\begin{Desc}
\item[Date]2007/09/03\end{Desc}
\end{Desc}
\begin{Desc}
\item[Author:]Noriaki Ando $<${\tt n-ando@aist.go.jp}$>$ and Shinji Kurihara\end{Desc}
Copyright (C) 2007 Task-intelligence Research Group, Intelligent Systems Research Institute, National Institute of Advanced Industrial Science and Technology (AIST), Japan All rights reserved.

\subsection{Function Documentation}
\index{CORBA_SeqUtil.py@{CORBA\_\-Seq\-Util.py}!back@{back}}
\index{back@{back}!CORBA_SeqUtil.py@{CORBA\_\-Seq\-Util.py}}
\subsubsection{\setlength{\rightskip}{0pt plus 5cm}back (seq)}\label{CORBA__SeqUtil_8py_a6}


Get the last element of the CORBA sequence. 

This operation returns seq[seq.length() - 1].

\begin{Desc}
\item[Parameters:]
\begin{description}
\item[{\em seq}]The CORBA sequence to be get the element\end{description}
\end{Desc}
\index{CORBA_SeqUtil.py@{CORBA\_\-Seq\-Util.py}!clear@{clear}}
\index{clear@{clear}!CORBA_SeqUtil.py@{CORBA\_\-Seq\-Util.py}}
\subsubsection{\setlength{\rightskip}{0pt plus 5cm}clear (seq)}\label{CORBA__SeqUtil_8py_a9}


Erase all the elements of the CORBA sequence. 

same as seq.length(0).\index{CORBA_SeqUtil.py@{CORBA\_\-Seq\-Util.py}!erase@{erase}}
\index{erase@{erase}!CORBA_SeqUtil.py@{CORBA\_\-Seq\-Util.py}}
\subsubsection{\setlength{\rightskip}{0pt plus 5cm}erase (seq, index)}\label{CORBA__SeqUtil_8py_a7}


Erase the element of the specified index. 

This operation removes the element of the given index. The other elements are closed up around the hole.

\begin{Desc}
\item[Parameters:]
\begin{description}
\item[{\em seq}]The CORBA sequence to be get the element \item[{\em index}]The index of the element to be removed\end{description}
\end{Desc}
\index{CORBA_SeqUtil.py@{CORBA\_\-Seq\-Util.py}!erase_if@{erase\_\-if}}
\index{erase_if@{erase\_\-if}!CORBA_SeqUtil.py@{CORBA\_\-Seq\-Util.py}}
\subsubsection{\setlength{\rightskip}{0pt plus 5cm}erase\_\-if (seq, f)}\label{CORBA__SeqUtil_8py_a8}


\index{CORBA_SeqUtil.py@{CORBA\_\-Seq\-Util.py}!find@{find}}
\index{find@{find}!CORBA_SeqUtil.py@{CORBA\_\-Seq\-Util.py}}
\subsubsection{\setlength{\rightskip}{0pt plus 5cm}find (seq, f)}\label{CORBA__SeqUtil_8py_a1}


Return the index of CORBA sequence element that functor matches. 

This operation applies the given functor to the given CORBA sequence, and returns the index of the sequence element that the functor matches. The functor should be bool functor(const CORBA sequence element) type, and it would return true, if the element matched the functor.

\begin{Desc}
\item[Returns:]The index of the element that functor matches. If no element found, it would return -1. \end{Desc}
\begin{Desc}
\item[Parameters:]
\begin{description}
\item[{\em seq}]CORBA sequence to be applied the functor \item[{\em functor}]A functor to process CORBA sequence elements\end{description}
\end{Desc}
\index{CORBA_SeqUtil.py@{CORBA\_\-Seq\-Util.py}!for_each@{for\_\-each}}
\index{for_each@{for\_\-each}!CORBA_SeqUtil.py@{CORBA\_\-Seq\-Util.py}}
\subsubsection{\setlength{\rightskip}{0pt plus 5cm}for\_\-each (seq, f)}\label{CORBA__SeqUtil_8py_a0}


CORBA sequence helper template functions\-Apply the functor to all CORBA sequence elements. 

Apply the given functor to the given CORBA sequence. functor should be void functor(CORBA sequence element).

\begin{Desc}
\item[Returns:]Functor that processed all CORBA sequence elements \end{Desc}
\begin{Desc}
\item[Parameters:]
\begin{description}
\item[{\em seq}]CORBA sequence to be applied the functor \item[{\em functor}]A functor to process CORBA sequence elements\end{description}
\end{Desc}
\index{CORBA_SeqUtil.py@{CORBA\_\-Seq\-Util.py}!front@{front}}
\index{front@{front}!CORBA_SeqUtil.py@{CORBA\_\-Seq\-Util.py}}
\subsubsection{\setlength{\rightskip}{0pt plus 5cm}front (seq)}\label{CORBA__SeqUtil_8py_a5}


Get the front element of the CORBA sequence. 

This operation returns seq[0].

\begin{Desc}
\item[Parameters:]
\begin{description}
\item[{\em seq}]The CORBA sequence to be get the element\end{description}
\end{Desc}
\index{CORBA_SeqUtil.py@{CORBA\_\-Seq\-Util.py}!insert@{insert}}
\index{insert@{insert}!CORBA_SeqUtil.py@{CORBA\_\-Seq\-Util.py}}
\subsubsection{\setlength{\rightskip}{0pt plus 5cm}insert (seq, elem, index)}\label{CORBA__SeqUtil_8py_a4}


Insert the element to the CORBA sequence. 

Insert a new element in the given position to the CORBA sequence. If the given index is greater than the length of the sequence, the given element is pushed back to the last of the sequence. The length of the CORBA sequence will be expanded automatically.

\begin{Desc}
\item[Parameters:]
\begin{description}
\item[{\em seq}]The CORBA sequence to be inserted a new element \item[{\em elem}]The new element to be inserted the sequence \item[{\em index}]The inserting position\end{description}
\end{Desc}
\index{CORBA_SeqUtil.py@{CORBA\_\-Seq\-Util.py}!push_back@{push\_\-back}}
\index{push_back@{push\_\-back}!CORBA_SeqUtil.py@{CORBA\_\-Seq\-Util.py}}
\subsubsection{\setlength{\rightskip}{0pt plus 5cm}push\_\-back (seq, elem)}\label{CORBA__SeqUtil_8py_a2}


Push the new element back to the CORBA sequence. 

Add the given element to the last of CORBA sequence. The length of the CORBA sequence will be expanded automatically.

\begin{Desc}
\item[Parameters:]
\begin{description}
\item[{\em seq}]CORBA sequence to be added a new element \item[{\em elem}]The new element to be added to the CORBA sequence\end{description}
\end{Desc}
\index{CORBA_SeqUtil.py@{CORBA\_\-Seq\-Util.py}!push_back_list@{push\_\-back\_\-list}}
\index{push_back_list@{push\_\-back\_\-list}!CORBA_SeqUtil.py@{CORBA\_\-Seq\-Util.py}}
\subsubsection{\setlength{\rightskip}{0pt plus 5cm}push\_\-back\_\-list (seq1, seq2)}\label{CORBA__SeqUtil_8py_a3}


