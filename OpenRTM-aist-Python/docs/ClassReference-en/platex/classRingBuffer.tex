\section{Ring\-Buffer Class Reference}
\label{classRingBuffer}\index{RingBuffer@{RingBuffer}}
\subsection*{Public Member Functions}
\begin{CompactItemize}
\item 
{\bf \_\-\_\-init\_\-\_\-} (length)
\item 
{\bf \_\-\_\-del\_\-\_\-} ()
\begin{CompactList}\small\item\em virtual destractor \item\end{CompactList}\item 
{\bf init} (data)
\item 
{\bf clear} ()
\item 
{\bf length} ()
\begin{CompactList}\small\item\em Get the buffer length. \item\end{CompactList}\item 
{\bf write} (value)
\begin{CompactList}\small\item\em Write data into the buffer. \item\end{CompactList}\item 
{\bf read} (value)
\begin{CompactList}\small\item\em Read data from the buffer. \item\end{CompactList}\item 
{\bf is\-Full} ()
\begin{CompactList}\small\item\em True if the buffer is full, else false. \item\end{CompactList}\item 
{\bf is\-Empty} ()
\begin{CompactList}\small\item\em True if the buffer is empty, else false. \item\end{CompactList}\item 
{\bf is\-New} ()
\item 
{\bf put} (data)
\begin{CompactList}\small\item\em Write data into the buffer. \item\end{CompactList}\item 
{\bf get} ()
\begin{CompactList}\small\item\em Get data from the buffer. \item\end{CompactList}\item 
{\bf get\-Ref} ()
\begin{CompactList}\small\item\em Get the buffer's reference to be written the next. \item\end{CompactList}\end{CompactItemize}
\subsection*{Classes}
\begin{CompactItemize}
\item 
class {\bf Data}
\begin{CompactList}\small\item\em Buffer sequence. \item\end{CompactList}\end{CompactItemize}


\subsection{Member Function Documentation}
\index{RingBuffer@{Ring\-Buffer}!__del__@{\_\-\_\-del\_\-\_\-}}
\index{__del__@{\_\-\_\-del\_\-\_\-}!RingBuffer@{Ring\-Buffer}}
\subsubsection{\setlength{\rightskip}{0pt plus 5cm}Ring\-Buffer::\_\-\_\-del\_\-\_\- ()}\label{classRingBuffer_RingBuffera1}


virtual destractor 

\index{RingBuffer@{Ring\-Buffer}!__init__@{\_\-\_\-init\_\-\_\-}}
\index{__init__@{\_\-\_\-init\_\-\_\-}!RingBuffer@{Ring\-Buffer}}
\subsubsection{\setlength{\rightskip}{0pt plus 5cm}Ring\-Buffer::\_\-\_\-init\_\-\_\- (length)}\label{classRingBuffer_RingBuffera0}


\index{RingBuffer@{Ring\-Buffer}!clear@{clear}}
\index{clear@{clear}!RingBuffer@{Ring\-Buffer}}
\subsubsection{\setlength{\rightskip}{0pt plus 5cm}Ring\-Buffer::clear ()}\label{classRingBuffer_RingBuffera3}


\index{RingBuffer@{Ring\-Buffer}!get@{get}}
\index{get@{get}!RingBuffer@{Ring\-Buffer}}
\subsubsection{\setlength{\rightskip}{0pt plus 5cm}Ring\-Buffer::get ()}\label{classRingBuffer_RingBuffera11}


Get data from the buffer. 

\index{RingBuffer@{Ring\-Buffer}!getRef@{getRef}}
\index{getRef@{getRef}!RingBuffer@{Ring\-Buffer}}
\subsubsection{\setlength{\rightskip}{0pt plus 5cm}Ring\-Buffer::get\-Ref ()}\label{classRingBuffer_RingBuffera12}


Get the buffer's reference to be written the next. 

\index{RingBuffer@{Ring\-Buffer}!init@{init}}
\index{init@{init}!RingBuffer@{Ring\-Buffer}}
\subsubsection{\setlength{\rightskip}{0pt plus 5cm}Ring\-Buffer::init (data)}\label{classRingBuffer_RingBuffera2}


\index{RingBuffer@{Ring\-Buffer}!isEmpty@{isEmpty}}
\index{isEmpty@{isEmpty}!RingBuffer@{Ring\-Buffer}}
\subsubsection{\setlength{\rightskip}{0pt plus 5cm}Ring\-Buffer::is\-Empty ()}\label{classRingBuffer_RingBuffera8}


True if the buffer is empty, else false. 

\index{RingBuffer@{Ring\-Buffer}!isFull@{isFull}}
\index{isFull@{isFull}!RingBuffer@{Ring\-Buffer}}
\subsubsection{\setlength{\rightskip}{0pt plus 5cm}Ring\-Buffer::is\-Full ()}\label{classRingBuffer_RingBuffera7}


True if the buffer is full, else false. 

\index{RingBuffer@{Ring\-Buffer}!isNew@{isNew}}
\index{isNew@{isNew}!RingBuffer@{Ring\-Buffer}}
\subsubsection{\setlength{\rightskip}{0pt plus 5cm}Ring\-Buffer::is\-New ()}\label{classRingBuffer_RingBuffera9}


\index{RingBuffer@{Ring\-Buffer}!length@{length}}
\index{length@{length}!RingBuffer@{Ring\-Buffer}}
\subsubsection{\setlength{\rightskip}{0pt plus 5cm}Ring\-Buffer::length ()}\label{classRingBuffer_RingBuffera4}


Get the buffer length. 

\index{RingBuffer@{Ring\-Buffer}!put@{put}}
\index{put@{put}!RingBuffer@{Ring\-Buffer}}
\subsubsection{\setlength{\rightskip}{0pt plus 5cm}Ring\-Buffer::put (data)}\label{classRingBuffer_RingBuffera10}


Write data into the buffer. 

\index{RingBuffer@{Ring\-Buffer}!read@{read}}
\index{read@{read}!RingBuffer@{Ring\-Buffer}}
\subsubsection{\setlength{\rightskip}{0pt plus 5cm}Ring\-Buffer::read (value)}\label{classRingBuffer_RingBuffera6}


Read data from the buffer. 

\index{RingBuffer@{Ring\-Buffer}!write@{write}}
\index{write@{write}!RingBuffer@{Ring\-Buffer}}
\subsubsection{\setlength{\rightskip}{0pt plus 5cm}Ring\-Buffer::write (value)}\label{classRingBuffer_RingBuffera5}


Write data into the buffer. 



The documentation for this class was generated from the following file:\begin{CompactItemize}
\item 
{\bf Ring\-Buffer.py}\end{CompactItemize}
