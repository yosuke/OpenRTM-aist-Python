\section{Properties Class Reference}
\label{classProperties}\index{Properties@{Properties}}
\subsection*{Public Member Functions}
\begin{CompactItemize}
\item 
{\bf \_\-\_\-init\_\-\_\-} (key=None, value=None, defaults\_\-map=None, defaults\_\-str=None, num=None, prop=None)
\begin{CompactList}\small\item\em Constructor. \item\end{CompactList}\item 
{\bf assigment\-Operator} (prop)
\begin{CompactList}\small\item\em Assignment operator. \item\end{CompactList}\item 
{\bf \_\-\_\-del\_\-\_\-} ()
\begin{CompactList}\small\item\em Destructor. \item\end{CompactList}\item 
{\bf get\-Name} ()
\item 
{\bf get\-Value} ()
\item 
{\bf get\-Default\-Value} ()
\item 
{\bf get\-Leaf} ()
\item 
{\bf get\-Root} ()
\item 
{\bf get\-Property} (key, default=None)
\begin{CompactList}\small\item\em Searches for the property with the specified key in this property. \item\end{CompactList}\item 
{\bf get\-Default} (key)
\begin{CompactList}\small\item\em Set value as the default value to specified key's property. \item\end{CompactList}\item 
{\bf set\-Property} (key, value=None)
\begin{CompactList}\small\item\em Sets a value associated with key in the property list. \item\end{CompactList}\item 
{\bf set\-Default} (key, value)
\begin{CompactList}\small\item\em Sets a default value associated with key in the property list. \item\end{CompactList}\item 
{\bf set\-Defaults} (defaults, num=None)
\begin{CompactList}\small\item\em Sets a default value associated with key in the property list. \item\end{CompactList}\item 
{\bf list} (out)
\begin{CompactList}\small\item\em Prints this property list out to the specified output stream. \item\end{CompactList}\item 
{\bf load} (in\-Stream)
\begin{CompactList}\small\item\em Loads property list consists of key:value from input stream. \item\end{CompactList}\item 
{\bf save} (out, header)
\begin{CompactList}\small\item\em Save the properties list to the stream. \item\end{CompactList}\item 
{\bf store} (out, header)
\begin{CompactList}\small\item\em Stores property list to the output stream. \item\end{CompactList}\item 
{\bf property\-Names} ()
\begin{CompactList}\small\item\em Returns an vector of all the keys in this property. \item\end{CompactList}\item 
{\bf size} ()
\begin{CompactList}\small\item\em Get number of Properties. \item\end{CompactList}\item 
{\bf get\-Node} (key)
\begin{CompactList}\small\item\em Get node of Properties. \item\end{CompactList}\item 
{\bf create\-Node} (key)
\item 
{\bf remove\-Node} (leaf\_\-name)
\begin{CompactList}\small\item\em Get node of Properties. \item\end{CompactList}\item 
{\bf has\-Key} (key)
\begin{CompactList}\small\item\em If key exists in the children. \item\end{CompactList}\item 
{\bf clear} ()
\begin{CompactList}\small\item\em If key exists in the children. \item\end{CompactList}\item 
{\bf merge\-Properties} (prop)
\begin{CompactList}\small\item\em Merge properties

C++148\AA{} Properties\& Properties::operator$<$$<$(const Properties\& prop)130\`{I}142\`{A}145149. \item\end{CompactList}\item 
{\bf split\-Key\-Value} (\_\-str, key, value)
\item 
{\bf split} (\_\-str, delim, value)
\item 
{\bf \_\-get\-Node} (keys, index, curr)
\item 
{\bf \_\-propertiy\-Names} (names, curr\_\-name, curr)
\item 
{\bf \_\-store} (out, curr\_\-name, curr)
\item 
{\bf indent} (index)
\item 
{\bf \_\-dump} (out, curr, index)
\item 
{\bf \_\-\_\-str\_\-\_\-} ()
\end{CompactItemize}


\subsection{Detailed Description}
The Properties class represents a persistent set of properties. The Properties can be saved to a stream or loaded from a stream. Each key and its corresponding value in the property list is a string.

A property list can contain another property list as its \char`\"{}defaults\char`\"{}; this second property list is searched if the property key is not found in the original property list.

Because Properties inherits from Hashtable, the put and put\-All methods can be applied to a Properties object. Their use is strongly discouraged as they allow the caller to insert entries whose keys or values are not Strings. The set\-Property method should be used instead. If the store or save method is called on a \char`\"{}compromised\char`\"{} Properties object that contains a non-String key or value, the call will fail.

The load and store methods load and store properties in a simple line-oriented format specified below. This format uses the ISO 8859-1 character encoding. Characters that cannot be directly represented in this encoding can be written using Unicode escapes ; only a single 'u' character is allowed in an escape sequence. The native2ascii tool can be used to convert property files to and from other character encodings.

This class has almost same methods of Java's Properties class. Input and Output stream of this properties are compatible each other except Unicode encoded property file.



\subsection{Member Function Documentation}
\index{Properties@{Properties}!__del__@{\_\-\_\-del\_\-\_\-}}
\index{__del__@{\_\-\_\-del\_\-\_\-}!Properties@{Properties}}
\subsubsection{\setlength{\rightskip}{0pt plus 5cm}Properties::\_\-\_\-del\_\-\_\- ()}\label{classProperties_Propertiesa2}


Destructor. 

\index{Properties@{Properties}!__init__@{\_\-\_\-init\_\-\_\-}}
\index{__init__@{\_\-\_\-init\_\-\_\-}!Properties@{Properties}}
\subsubsection{\setlength{\rightskip}{0pt plus 5cm}Properties::\_\-\_\-init\_\-\_\- (key = {\tt None}, value = {\tt None}, defaults\_\-map = {\tt None}, defaults\_\-str = {\tt None}, num = {\tt None}, prop = {\tt None})}\label{classProperties_Propertiesa0}


Constructor. 

Creates a root node of Property with root's key and value.\index{Properties@{Properties}!__str__@{\_\-\_\-str\_\-\_\-}}
\index{__str__@{\_\-\_\-str\_\-\_\-}!Properties@{Properties}}
\subsubsection{\setlength{\rightskip}{0pt plus 5cm}Properties::\_\-\_\-str\_\-\_\- ()}\label{classProperties_Propertiesa32}


friend std::ostream\& operator$<$$<$(std::ostream\& lhs, const Properties\& rhs); 130\`{I}145\~{a}130\'{\i}130\`{e}130\'{E}129Aprint obj130\'{E}130\"{A}140\"{A}130\~{N}143o130$\mu$137\^{A}148$\backslash$130\AE{}130$\mu$130\"{A}130162130\'{e}129B\index{Properties@{Properties}!_dump@{\_\-dump}}
\index{_dump@{\_\-dump}!Properties@{Properties}}
\subsubsection{\setlength{\rightskip}{0pt plus 5cm}Properties::\_\-dump (out, curr, index)}\label{classProperties_Propertiesa31}


\index{Properties@{Properties}!_getNode@{\_\-getNode}}
\index{_getNode@{\_\-getNode}!Properties@{Properties}}
\subsubsection{\setlength{\rightskip}{0pt plus 5cm}Properties::\_\-get\-Node (keys, index, curr)}\label{classProperties_Propertiesa27}


\index{Properties@{Properties}!_propertiyNames@{\_\-propertiyNames}}
\index{_propertiyNames@{\_\-propertiyNames}!Properties@{Properties}}
\subsubsection{\setlength{\rightskip}{0pt plus 5cm}Properties::\_\-propertiy\-Names (names, curr\_\-name, curr)}\label{classProperties_Propertiesa28}


\index{Properties@{Properties}!_store@{\_\-store}}
\index{_store@{\_\-store}!Properties@{Properties}}
\subsubsection{\setlength{\rightskip}{0pt plus 5cm}Properties::\_\-store (out, curr\_\-name, curr)}\label{classProperties_Propertiesa29}


\index{Properties@{Properties}!assigmentOperator@{assigmentOperator}}
\index{assigmentOperator@{assigmentOperator}!Properties@{Properties}}
\subsubsection{\setlength{\rightskip}{0pt plus 5cm}Properties::assigment\-Operator (prop)}\label{classProperties_Propertiesa1}


Assignment operator. 

\index{Properties@{Properties}!clear@{clear}}
\index{clear@{clear}!Properties@{Properties}}
\subsubsection{\setlength{\rightskip}{0pt plus 5cm}Properties::clear ()}\label{classProperties_Propertiesa23}


If key exists in the children. 

\index{Properties@{Properties}!createNode@{createNode}}
\index{createNode@{createNode}!Properties@{Properties}}
\subsubsection{\setlength{\rightskip}{0pt plus 5cm}Properties::create\-Node (key)}\label{classProperties_Propertiesa20}


\index{Properties@{Properties}!getDefault@{getDefault}}
\index{getDefault@{getDefault}!Properties@{Properties}}
\subsubsection{\setlength{\rightskip}{0pt plus 5cm}Properties::get\-Default (key)}\label{classProperties_Propertiesa9}


Set value as the default value to specified key's property. 

\index{Properties@{Properties}!getDefaultValue@{getDefaultValue}}
\index{getDefaultValue@{getDefaultValue}!Properties@{Properties}}
\subsubsection{\setlength{\rightskip}{0pt plus 5cm}Properties::get\-Default\-Value ()}\label{classProperties_Propertiesa5}


\index{Properties@{Properties}!getLeaf@{getLeaf}}
\index{getLeaf@{getLeaf}!Properties@{Properties}}
\subsubsection{\setlength{\rightskip}{0pt plus 5cm}Properties::get\-Leaf ()}\label{classProperties_Propertiesa6}


\index{Properties@{Properties}!getName@{getName}}
\index{getName@{getName}!Properties@{Properties}}
\subsubsection{\setlength{\rightskip}{0pt plus 5cm}Properties::get\-Name ()}\label{classProperties_Propertiesa3}


\index{Properties@{Properties}!getNode@{getNode}}
\index{getNode@{getNode}!Properties@{Properties}}
\subsubsection{\setlength{\rightskip}{0pt plus 5cm}Properties::get\-Node (key)}\label{classProperties_Propertiesa19}


Get node of Properties. 

\index{Properties@{Properties}!getProperty@{getProperty}}
\index{getProperty@{getProperty}!Properties@{Properties}}
\subsubsection{\setlength{\rightskip}{0pt plus 5cm}Properties::get\-Property (key, default = {\tt None})}\label{classProperties_Propertiesa8}


Searches for the property with the specified key in this property. 

Searches for the property with the specified key in this property list. If the key is not found in this property list, the default property list, and its defaults, recursively, are then checked. The method returns the default value argument if the property is not found.

\begin{Desc}
\item[Parameters:]
\begin{description}
\item[{\em key}]the property key \item[{\em default\-Value}]a default value. \end{description}
\end{Desc}
\begin{Desc}
\item[Returns:]the value in this property list with the specified key value.\end{Desc}
\index{Properties@{Properties}!getRoot@{getRoot}}
\index{getRoot@{getRoot}!Properties@{Properties}}
\subsubsection{\setlength{\rightskip}{0pt plus 5cm}Properties::get\-Root ()}\label{classProperties_Propertiesa7}


\index{Properties@{Properties}!getValue@{getValue}}
\index{getValue@{getValue}!Properties@{Properties}}
\subsubsection{\setlength{\rightskip}{0pt plus 5cm}Properties::get\-Value ()}\label{classProperties_Propertiesa4}


\index{Properties@{Properties}!hasKey@{hasKey}}
\index{hasKey@{hasKey}!Properties@{Properties}}
\subsubsection{\setlength{\rightskip}{0pt plus 5cm}Properties::has\-Key (key)}\label{classProperties_Propertiesa22}


If key exists in the children. 

\index{Properties@{Properties}!indent@{indent}}
\index{indent@{indent}!Properties@{Properties}}
\subsubsection{\setlength{\rightskip}{0pt plus 5cm}Properties::indent (index)}\label{classProperties_Propertiesa30}


\index{Properties@{Properties}!list@{list}}
\index{list@{list}!Properties@{Properties}}
\subsubsection{\setlength{\rightskip}{0pt plus 5cm}Properties::list (out)}\label{classProperties_Propertiesa13}


Prints this property list out to the specified output stream. 

Prints this property list out to the specified output stream. This method is useful for debugging.

\begin{Desc}
\item[Parameters:]
\begin{description}
\item[{\em out}]an output stream.\end{description}
\end{Desc}
\index{Properties@{Properties}!load@{load}}
\index{load@{load}!Properties@{Properties}}
\subsubsection{\setlength{\rightskip}{0pt plus 5cm}Properties::load (in\-Stream)}\label{classProperties_Propertiesa14}


Loads property list consists of key:value from input stream. 

Reads a property list (key and element pairs) from the input stream. The stream is assumed to be using the ISO 8859-1 character encoding; that is each byte is one Latin1 character. Characters not in Latin1, and certain special characters, can be represented in keys and elements using escape sequences similar to those used for character and string literals The differences from the character escape sequences used for characters and strings are:\begin{itemize}
\item Octal escapes are not recognized.\item The character sequence {\bf does} not represent a backspace character.\item The method does not treat a backslash character, $\backslash$, before a non-valid escape character as an error; the backslash is silently dropped. For example, in a Java string the sequence \char`\"{}$\backslash$z\char`\"{} would cause a compile time error. In contrast, this method silently drops the backslash. Therefore, this method treats the two character sequence \char`\"{}$\backslash$b\char`\"{} as equivalent to the single character 'b'.\item Escapes are not necessary for single and double quotes; however, by the rule above, single and double quote characters preceded by a backslash still yield single and double quote characters, respectively. An Illegal\-Argument\-Exception is thrown if a malformed Unicode escape appears in the input.\end{itemize}


This method processes input in terms of lines. A natural line of input is terminated either by a set of line terminator characters (\par
 or  or \par
) or by the end of the file. A natural line may be either a blank line, a comment line, or hold some part of a key-element pair. The logical line holding all the data for a key-element pair may be spread out across several adjacent natural lines by escaping the line terminator sequence with a backslash character, $\backslash$. Note that a comment line cannot be extended in this manner; every natural line that is a comment must have its own comment indicator, as described below. If a logical line is continued over several natural lines, the continuation lines receive further processing, also described below. Lines are read from the input stream until end of file is reached.

A natural line that contains only white space characters is considered blank and is ignored. A comment line has an ASCII ' ' or '!' as its first non-white space character; comment lines are also ignored and do not encode key-element information. In addition to line terminators, this method considers the characters space (' ', ''), tab ('', ''), and form feed ('', '') to be white space.

If a logical line is spread across several natural lines, the backslash escaping the line terminator sequence, the line terminator sequence, and any white space at the start the following line have no affect on the key or element values. The remainder of the discussion of key and element parsing will assume all the characters constituting the key and element appear on a single natural line after line continuation characters have been removed. Note that it is not sufficient to only examine the character preceding a line terminator sequence to see if the line terminator is escaped; there must be an odd number of contiguous backslashes for the line terminator to be escaped. Since the input is processed from left to right, a non-zero even number of 2n contiguous backslashes before a line terminator (or elsewhere) encodes n backslashes after escape processing.

The key contains all of the characters in the line starting with the first non-white space character and up to, but not including, the first unescaped '=', ':', or white space character other than a line terminator. All of these key termination characters may be included in the key by escaping them with a preceding backslash character; for example,

$\backslash$:$\backslash$=

would be the two-character key \char`\"{}:=\char`\"{}. Line terminator characters can be included using  and \par
 escape sequences. Any white space after the key is skipped; if the first non-white space character after the key is '=' or ':', then it is ignored and any white space characters after it are also skipped. All remaining characters on the line become part of the associated element string; if there are no remaining characters, the element is the empty string \char`\"{}\char`\"{}. Once the raw character sequences constituting the key and element are identified, escape processing is performed as described above.

As an example, each of the following three lines specifies the key \char`\"{}Truth\char`\"{} and the associated element value \char`\"{}Beauty\char`\"{}:

Truth = Beauty \par
 Truth:Beauty \par
 Truth :Beauty \par
 As another example, the following three lines specify a single property:

fruits apple, banana, pear, $\backslash$ \par
 cantaloupe, watermelon, $\backslash$ \par
 kiwi, mango \par
 The key is \char`\"{}fruits\char`\"{} and the associated element is: \char`\"{}apple, banana, pear, cantaloupe, watermelon, kiwi, mango\char`\"{}Note that a space appears before each $\backslash$ so that a space will appear after each comma in the final result; the $\backslash$, line terminator, and leading white space on the continuation line are merely discarded and are not replaced by one or more other characters. As a third example, the line:

cheeses \par
 specifies that the key is \char`\"{}cheeses\char`\"{} and the associated element is the empty string \char`\"{}\char`\"{}.

\begin{Desc}
\item[Parameters:]
\begin{description}
\item[{\em in\-Stream}]the input stream.\end{description}
\end{Desc}
\index{Properties@{Properties}!mergeProperties@{mergeProperties}}
\index{mergeProperties@{mergeProperties}!Properties@{Properties}}
\subsubsection{\setlength{\rightskip}{0pt plus 5cm}Properties::merge\-Properties (prop)}\label{classProperties_Propertiesa24}


Merge properties

C++148\AA{} Properties\& Properties::operator$<$$<$(const Properties\& prop)130\`{I}142\`{A}145149. 

\index{Properties@{Properties}!propertyNames@{propertyNames}}
\index{propertyNames@{propertyNames}!Properties@{Properties}}
\subsubsection{\setlength{\rightskip}{0pt plus 5cm}Properties::property\-Names ()}\label{classProperties_Propertiesa17}


Returns an vector of all the keys in this property. 

Returns an enumeration of all the keys in this property list, including distinct keys in the default property list if a key of the same name has not already been found from the main properties list.

\begin{Desc}
\item[Returns:]an vector of all the keys in this property list, including the keys in the default property list.\end{Desc}
\index{Properties@{Properties}!removeNode@{removeNode}}
\index{removeNode@{removeNode}!Properties@{Properties}}
\subsubsection{\setlength{\rightskip}{0pt plus 5cm}Properties::remove\-Node (leaf\_\-name)}\label{classProperties_Propertiesa21}


Get node of Properties. 

\index{Properties@{Properties}!save@{save}}
\index{save@{save}!Properties@{Properties}}
\subsubsection{\setlength{\rightskip}{0pt plus 5cm}Properties::save (out, header)}\label{classProperties_Propertiesa15}


Save the properties list to the stream. 

Deprecated.

\begin{Desc}
\item[Parameters:]
\begin{description}
\item[{\em out}]The output stream \item[{\em header}]A description of the property list\end{description}
\end{Desc}
\index{Properties@{Properties}!setDefault@{setDefault}}
\index{setDefault@{setDefault}!Properties@{Properties}}
\subsubsection{\setlength{\rightskip}{0pt plus 5cm}Properties::set\-Default (key, value)}\label{classProperties_Propertiesa11}


Sets a default value associated with key in the property list. 

\index{Properties@{Properties}!setDefaults@{setDefaults}}
\index{setDefaults@{setDefaults}!Properties@{Properties}}
\subsubsection{\setlength{\rightskip}{0pt plus 5cm}Properties::set\-Defaults (defaults, num = {\tt None})}\label{classProperties_Propertiesa12}


Sets a default value associated with key in the property list. 

\index{Properties@{Properties}!setProperty@{setProperty}}
\index{setProperty@{setProperty}!Properties@{Properties}}
\subsubsection{\setlength{\rightskip}{0pt plus 5cm}Properties::set\-Property (key, value = {\tt None})}\label{classProperties_Propertiesa10}


Sets a value associated with key in the property list. 

This method sets the \char`\"{}value\char`\"{} associated with \char`\"{}key\char`\"{} in the property list. If the property list has a value of \char`\"{}key\char`\"{}, old value is returned.

\begin{Desc}
\item[Parameters:]
\begin{description}
\item[{\em key}]the key to be placed into this property list. \item[{\em value}]the value corresponding to key. \end{description}
\end{Desc}
\begin{Desc}
\item[Returns:]the previous value of the specified key in this property list, or null if it did not have one.\end{Desc}
\index{Properties@{Properties}!size@{size}}
\index{size@{size}!Properties@{Properties}}
\subsubsection{\setlength{\rightskip}{0pt plus 5cm}Properties::size ()}\label{classProperties_Propertiesa18}


Get number of Properties. 

\index{Properties@{Properties}!split@{split}}
\index{split@{split}!Properties@{Properties}}
\subsubsection{\setlength{\rightskip}{0pt plus 5cm}Properties::split (\_\-str, delim, value)}\label{classProperties_Propertiesa26}


\index{Properties@{Properties}!splitKeyValue@{splitKeyValue}}
\index{splitKeyValue@{splitKeyValue}!Properties@{Properties}}
\subsubsection{\setlength{\rightskip}{0pt plus 5cm}Properties::split\-Key\-Value (\_\-str, key, value)}\label{classProperties_Propertiesa25}


\index{Properties@{Properties}!store@{store}}
\index{store@{store}!Properties@{Properties}}
\subsubsection{\setlength{\rightskip}{0pt plus 5cm}Properties::store (out, header)}\label{classProperties_Propertiesa16}


Stores property list to the output stream. 

Writes this property list (key and element pairs) in this Properties table to the output stream in a format suitable for loading into a Properties table using the load method. The stream is written using the ISO 8859-1 character encoding.

Properties from the defaults table of this Properties table (if any) are not written out by this method.

If the comments argument is not null, then an ASCII character, the comments string, and a line separator are first written to the output stream. Thus, the comments can serve as an identifying comment.

Next, a comment line is always written, consisting of an ASCII character, the current date and time (as if produced by the to\-String method of Date for the current time), and a line separator as generated by the Writer.

Then every entry in this Properties table is written out, one per line. For each entry the key string is written, then an ASCII =, then the associated element string. Each character of the key and element strings is examined to see whether it should be rendered as an escape sequence. The ASCII characters $\backslash$, tab, form feed, newline, and carriage return are written as $\backslash$, ,  \par
, and , respectively. Characters less than  and characters greater than  are written as  for the appropriate hexadecimal value xxxx. For the key, all space characters are written with a preceding $\backslash$ character. For the element, leading space characters, but not embedded or trailing space characters, are written with a preceding $\backslash$ character. The key and element characters , !, =, and : are written with a preceding backslash to ensure that they are properly loaded.

After the entries have been written, the output stream is flushed. The output stream remains open after this method returns.

\begin{Desc}
\item[Parameters:]
\begin{description}
\item[{\em out}]an output stream. \item[{\em header}]a description of the property list.\end{description}
\end{Desc}


The documentation for this class was generated from the following file:\begin{CompactItemize}
\item 
{\bf Properties.py}\end{CompactItemize}
